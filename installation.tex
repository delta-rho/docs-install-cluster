\section{Tessera Stack Cluster Installation Guide}
\subsection{Introduction}
This is a guide to installing the Tessera stack consisting of
Hadoop\textregistered, RHIPE, datadr, trelliscope and other supporting
packages on a multi-node cluster.  This guide is intended to be used
by the Systems Administrator to quickly install and configure the
software necessary to run the full Tessera stack.  It is not focused
on tuning and optimally configuring the Tessera stack.  It simplifies
and generalizes certain aspects of the installation.  The Systems
Administrator is encouraged to consult individual package
documentation for further tuning later.  In this guide we use assume
Ubuntu\textregistered Linux version 14.04 is the operating system
installed on he cluster.  A Red Hat Enterprise Linux version of this
installation guide is available at
\url{http://tessera.io/docs-install-cluster/install-redhat.html}.


\subsection{Example Configuration}
Running the Tessera stack on a Hadoop cluster requires each Hadoop node to be configured to fill one or more of the roles of ResourceManager, NameNode, Secondary NameNode, DataNode, and NodeManager. The role descriptions are as follows.

\begin{itemize}
\item NameNode - Manages the directory tree and metadata of the Hadoop File System (HDFS), a distributed virtual file system that allows data to be replicated and stored across many nodes in the cluster. 
\item SecondaryNameNode - Offloads HDFS checkpoint support for the NameNode.  It is not a NameNode failover or backup as the name may imply.
\item ResourceManager - Schedules and issues map reduce jobs for NodeManagers across the cluster.
\item DataNode - Stores data on the local drives of nodes as part of the distributed HDFS.  DataNodes are almost always also NodeManagers.
\item NodeManager - Executes the map and reduce jobs issued by the ResourceManager. NodeManagers are almost always also DataNodes.
\end{itemize}

There can be just one NameNode, one ResourceManager, and one Secondary
NameNode, but there will be many DataNodes and NodeManagers.  The node
acting as NameNode cannot also be the Secondary NameNode, but could
perform one or more of other roles of ResourceManager, DataNode, and
NodeManager with a potential  performance penalty.  Most cluster nodes
will be compute nodes with multiple hard drives and thus will run both
the DataNode and NodeManager services.

For our example we will have a single front-end server that acts as the
R-session server.  It is where we will run R, which in turn runs RHIPE.
The front-end server is aware of our Hadoop configuration and can issue
Hadoop commands, though it isn't technically a Hadoop node itself. The
front-end server will also be our Shiny server and optional Rstudio
server.  We will have five nodes in our example Hadoop cluster.  Each of
the Hadoop nodes in the cluster has a fully qualified domain
name of the form \verb|nodeNNN.example.com|. In our example,

\begin{itemize}
\item \verb|node001.example.com| will be the NameNode and the ResourceManager.
\item \verb|node002.example.com| will be the Secondary NameNode, a NodeManager, and a DataNode. 
\item \verb|node003.example.com| to \verb|node005.example.com| will be NodeManagers and DataNodes. 
\item \verb|frontend.example.com| will be the front end R-session server, the Shiny server, and optional Rstudio Server.
\end{itemize}

\subsection{Packages}
\subsubsection{Required Packages}
The Tessera stack consists of some packages provided by the Tessera group and some provided by other sources. \\

\textbf{Packages provided by the Tessera group}
\begin{enumerate}
\item RHIPE - 0.75
\item datadr
\item trelliscope
\end{enumerate}
\textbf{Packages provided by others}
\begin{enumerate}
\item pkg-config
\item Sun/Oracle Java - 7
\item Protocol Buffers - 2.5.0 
\item Cloudera Hadoop version cdh5.3.2
\item R - 3.0.1
\item rJava - 0.9-6 
\item OpenSSL (latest stable version)
\item R packages - codetools, MASS, ggplot2, lattice, boot, shiny, devtools
\item gdebi
\item Shiny server
\end{enumerate}

\subsubsection{Optional Packages}
These optional packages can be installed along with the Tessera stack providing convenience for the analyst. 
\begin{enumerate}
\item Rstudio server - Rstudio server creates a persistent interactive R session for the analyst by running R on a remote server and providing access through a the analyst's local browser.
\end{enumerate}

\newpage

\subsection{Installation}
Some packages need only be installed on the Hadoop nodes, some need only
be installed on the frontend server, and some must be installed on both.
This is specified on a per-package basis below. This guide is command
line friendly.


\subsubsection{System Update (all servers)}
First update all currently installed system packages.

\begin{verbatim}
sudo apt-get update
sudo apt-get upgrade
\end{verbatim}

\subsubsection{pkg-config (all servers)} (\url{http://www.freedesktop.org/wiki/Software/pkg-config/})\\
Following the system update, pkg-config is installed.
\begin{verbatim}
sudo apt-get install pkg-config
\end{verbatim}

\subsubsection{Java (all servers)} (\url{http://www.java.com/en/})\\
A Java repository for Sun/Oracle Java is first added to the Ubuntu APT package management system.
\begin{verbatim}
sudo apt-get install python-software-properties
sudo add-apt-repository ppa:webupd8team/java
\end{verbatim}
The system repository store is updated and Java is installed.
\begin{verbatim}
sudo apt-get update
sudo apt-get install oracle-java7-installer
\end{verbatim}

%%%%%%%%%%%%%%%%%% Don't do this?
%Java environment variables are added to the user environment for future logins.
%\begin{verbatim}
%echo "export JAVA_HOME=/usr/lib/jvm/java-7-oracle
%export JAVA_BIN=\$JAVA_HOME/bin
%export PATH=\$PATH:\$JAVA_HOME:\$JAVA_BIN" |
%sudo bash -c "cat > /etc/profile.d/java.sh"
%\end{verbatim}
%%%%%%

\subsubsection{Protocol Buffers (all servers)} (\url{https://code.google.com/p/protobuf/})\\
Install Protocol Buffers version 2.5.0.  It is important that version
2.5.0 be installed and not a newer or older version.  Ubuntu 14.04 should
have version 2.5.0 available by default.  Confirm this first by installing the
\verb|aptitude| package manager and ensuring it is version 2.5.0:

\begin{verbatim}
sudo apt-get install aptitude
aptitude show protobuf-compiler
\end{verbatim}

If it says it is version 2.5.0-something, then proceed to install it via:

\begin{verbatim}
sudo apt-get install protobuf-compiler protobuf-c-compiler libprotobuf-dev
\end{verbatim}

If it does NOT say it is version 2.5.0-something, then manually install
it.  First cd to a directory the software can be downloaded and built,
and type the following:

\begin{verbatim}
# Only install manually if the steps above did not say version 2.5.0-something
sudo apt-get install libtool
wget https://github.com/google/protobuf/archive/v2.5.0.zip
unzip v2.5.0.zip
cd protobuf-2.5.0
./autogen.sh 
./configure --prefix=/usr/local
make
sudo make install
\end{verbatim}


\subsubsection{Hadoop (all servers)} (\url{http://www.cloudera.com/content/cloudera/en/products-and-services/cdh.html})\\
We use the Hadoop distribution built by Cloudera.  It is a Hadoop MRv2 /
YARN implementation.  Download the Cloudera repository update package and
install it.  This will add the Cloudera package repository and repository
key to the Ubuntu repositories searched when installing packages:

\begin{verbatim}
wget http://archive.cloudera.com/cdh5/one-click-install/trusty/\
amd64/cdh5-repository_1.0_all.deb
sudo dpkg -i cdh5-repository_1.0_all.deb
sudo apt-get update
\end{verbatim}

We install different packages on each node based on the Hadoop roles
taken by that node.  Note that when these installs complete you will see
warnings such as "Failed to start Hadoop namenode. Return value: 1".
This is because we have not yet completed the configuration of these
packages so the initial attempt to launch them will fail.

node001.example.com:
\begin{verbatim}
sudo apt-get install hadoop-yarn-resourcemanager hadoop-hdfs-namenode \
hadoop-client
\end{verbatim}

node002.example.com:
\begin{verbatim}
sudo apt-get install hadoop-yarn-nodemanager hadoop-hdfs-secondarynamenode \
hadoop-hdfs-datanode hadoop-mapreduce 
\end{verbatim}

node003.example.com through node005.example.com:
\begin{verbatim}
sudo apt-get install hadoop-yarn-nodemanager hadoop-hdfs-datanode \
hadoop-mapreduce
\end{verbatim}

frontend.example.com (doesn't run any hadoop services, but needs the hadoop commands):
\begin{verbatim}
sudo apt-get install hadoop-client
\end{verbatim}

As part of the above steps, three new users and groups named yarn, hdfs,
and mapred will be added to the system.  All three users are added to
the new group hadoop.

Configuration of Hadoop settings is done in another section, but for now
we will prepare the configuration directory and set it as the default
Hadoop configuration on the system.  On each of node001 through node005,
and on the frontend execute the following.  We choose the configuration
directory name of "conf.tessera" in this guide:

\begin{verbatim}
sudo cp -r /etc/hadoop/conf.empty /etc/hadoop/conf.tessera
sudo update-alternatives --install /etc/hadoop/conf hadoop-conf \
/etc/hadoop/conf.tessera 50
sudo update-alternatives --set hadoop-conf /etc/hadoop/conf.tessera
\end{verbatim}


Add Hadoop environment variables and an updated path to the user environment for future logins.  This assumes users use an sh/ksh/bash/zsh based login shell.
\begin{verbatim}
echo "export HADOOP=/usr/lib/hadoop
export HADOOP_HOME=\$HADOOP	
export HADOOP_BIN=\$HADOOP/bin
export HADOOP_LIB=\$HADOOP/lib
export HADOOP_LIBS=\`hadoop classpath | tr -d \\*\`
export HADOOP_CONF_DIR=/etc/hadoop/conf.tessera
export HADOOP_COMMON_HOME=/usr/lib/hadoop
export HADOOP_MAPRED_HOME=/usr/lib/hadoop-mapreduce
export HADOOP_HDFS_HOME=/usr/lib/hadoop-hdfs
export YARN_HOME=/usr/lib/hadoop-yarn
export PATH=\$PATH:\$HADOOP_BIN" | 
sudo bash -c "cat > /etc/profile.d/hadoop.sh"
\end{verbatim}

\subsubsection{R (all nodes)} (\url{http://cran.r-project.org/})\\
R is installed and configured to know about Java.
\begin{verbatim}
sudo apt-get install r-base r-base-dev r-recommended r-cran-rodbc
sudo R CMD javareconf
\end{verbatim}

\subsubsection{rJava (all nodes)}(\url{http://www.rforge.net/rJava/})\\
rJava is also required by RHIPE.  It is installed by:

\begin{verbatim}
sudo apt-get install r-cran-rjava
\end{verbatim}

\subsubsection{RHIPE (all nodes)}(\url{http://www.tessera.io/})\\
RHIPE is downloaded and installed.
\begin{verbatim}
wget http://ml.stat.purdue.edu/rhipebin/Rhipe_0.75.0_cdh5mr2.tar.gz
sudo R CMD INSTALL Rhipe_0.75.0_cdh5mr2.tar.gz
\end{verbatim}

\subsubsection{OpenSSL (all nodes)}(\url{https://www.openssl.org/})\\
OpenSSL package is required by the Github installer/package handler for R. 
\begin{verbatim}
sudo apt-get install libcurl4-openssl-dev
\end{verbatim}

\subsubsection{datadr and trelliscope support packages (all nodes)}(\url{http://cran.r-project.org/web/packages/available\_packages\_by\_name.html})\\
R packages - codetools, latttice, MASS, ggplot2, boot, shiny and devtools are needed for trelliscope and datadr. They are installed using the R internal package installer \textbf{install.packages}. Use the command line to enter the following commands. 
\begin{verbatim}
sudo apt-get install r-cran-codetools r-cran-mass r-cran-ggplot2 \
r-cran-lattice r-cran-boot
sudo R -e "install.packages('RCurl', repos='http://cran.rstudio.com/')"
sudo R -e "install.packages('shiny', repos='http://cran.rstudio.com/')"
sudo R -e "install.packages('devtools', repos='http://cran.rstudio.com/')"
\end{verbatim}

\subsubsection{datadr and trelliscope (all nodes)}(\url{http://www.tessera.io})\\
datadr and trelliscope are installed using install\_github package from R. 
\begin{verbatim}
sudo R -e "options(repos = 'http://cran.rstudio.com/');\
 library(devtools); install_github('tesseradata/datadr')"
sudo R -e "options(repos = 'http://cran.rstudio.com/');\
 library(devtools); install_github('tesseradata/trelliscope')"
\end{verbatim}

\subsubsection{gdebi-core (frontend server)}(\url{https://apps.ubuntu.com/cat/applications/precise/gdebi/})\\
gdebi is a deb package installer. It automatically resolves and installs library dependencies.
\begin{verbatim}
sudo apt-get install gdebi-core
\end{verbatim}

\subsubsection{Shiny server (frontend server)}(\url{http://www.rstudio.com/products/shiny/shiny-server/})\\

Shiny server is only installed on frontend.example.com, our designated Shiny server.

\begin{verbatim}
sudo R -e "install.packages('shiny', repos='http://cran.rstudio.com/')"
wget http://download3.rstudio.org/ubuntu-12.04/x86_64/shiny-server-1.3.0.403-amd64.deb
sudo gdebi --n shiny-server-1.3.0.403-amd64.deb
\end{verbatim}

\subsubsection{Rstudio Server (frontend server)}(\url{http://www.rstudio.com/})\\
Rstudio Server is installed on just the shiny server, frontend.example.com.
\begin{verbatim}
wget http://download2.rstudio.org/rstudio-server-0.98.1103-amd64.deb \
-O /tmp/rstudio-server.deb
sudo gdebi --n /tmp/rstudio-server.deb
rm /tmp/rstudio-server.deb
\end{verbatim}

Rstudio Server should now be running on frontend.example.com.  Connect to
it by browsing to \url{http://frontend.example.com:8787} and logging in as a
normal user with accounts on that server.  This may require firewall changes.
