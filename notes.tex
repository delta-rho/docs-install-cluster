\subsection{Notes}

\subsubsection{RAID and Redundancy Design under Hadoop/HDFS}
\begin{itemize}
\item RAID configurations are usually not recommended for HDFS data drives.
HDFS already handles fault tolerance by distributing the blocks it writes
to local drives among all nodes for both performance and redundancy.
RAID won't improve performance, and could even slow things down.
In some configurations it will reduce overall HDFS capacity.
\item The default block redundancy setting for HDFS is
three replicates, as specified by the \verb|dfs.replication| variable
in \verb|hdfs-site.xml|. This means that each data block is copied to
three drives, optimally on three different nodes. Hadoop has shown high
availability is possible with three replicates.  The downside is the total
capacity of HDFS is divided by the number of replicates used.  This means
our 32 TB example cluster with three replicates can only hold 10.67 TB
of data.  Decreasing \verb|dfs.replication| below 3 is not recommended
and should be avoided.  Increasing it above 3 could increase performance for
large clusters under certain workloads, but at the cost of capacity.
\end{itemize}

\subsubsection{R Package Design}
\begin{itemize}
\item Most R packages can be completely provided by the system
administrator by installing them as root, which implicitly places them in
a system wide accessible location, for example \verb|/usr/lib/R/library|
or \verb|/usr/local/lib64/R/library|.
\item Alternately, the system administrator can install just the core
default R packages at a system wide location and allow individual
users to install specific R library packages in their home directory.
This permits users the flexibility to easily change versions of packages
they are using and update them when they choose.
\end{itemize} 

\subsubsection{RHIPE\_HADOOP\_TMP\_FOLDER environment variable}
\begin{itemize}
\item It has been observed that some versions of Linux, such as Red
Hat Enterprise Linux, may have an issue with RHIPE where it will give
false errors about being unable to write files in HDFS, even where
the directory in question is clearly writable.  This can be corrected
by creating a directory somewhere in HDFS that is readable only by
that user, and then setting the RHIPE\_HADOOP\_TMP\_FOLDER environment
variable to point to that HDFS directory.  The user bob for example,
would type this on frontend:

\begin{verbatim}
hadoop fs -mkdir -p /user/bob/tmp
hadoop fs -chmod 700 /user/bob/tmp
\end{verbatim}

He would then add this to his environment by including it in his .bashrc or .bash\_profile, or whatever location is appropriate for his shell:

\begin{verbatim}
export RHIPE_HADOOP_TMP_FOLDER=/user/bob/tmp
\end{verbatim}
\end{itemize}

\subsubsection{Consider using HDFS High Availability rather than a Secondary NameNode}
The primary role of the Secondary NameNode is to perform periodic
checkpointing so the NameNode doesn't have to, which makes reboots of
the cluster go much more quickly.  The Secondary NameNode could also be
used to reconstruct the majority of HDFS if the NameNode were to have
a catastrophic failure, but through a manual, imperfect process prone
to error.  For a more robust, fault tolerant Hadoop configuration,
consider using a High Availability HDFS configuration, which uses
a Standby NameNode rather than a Secondary NameNode, and can be
configured for automatic failover in the event of a NameNode failure.
This configuration is more complex, requires the use of Zookeeper, three
or more JournalNode hosts (these can be regular nodes), and another
node dedicated to act as the Standby NameNode.  The documentation at
cloudera.com describes this in more detail.
