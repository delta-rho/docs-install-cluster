\subsection{Hadoop Pre-run Setup}
At this point the configuration files created in
\verb|/etc/hadoop/conf.tessera| on node001 should be copied to all
other nodes.  Any future changes to any configuration files should be
done on node001 and then copied from there to all other nodes.

Note: In the event that some nodes have different number of drives,
or the paths to those drives differ, or they have a different number
of CPUs, different amounts of RAM, etc. then separate, independent
\verb|hdfs-site.xml| and \verb|yarn-site.xml| files may be necessary on
each node.

Next we format the new HDFS filesystem before actually starting the
NameNode service for the first time.  This is done on the NameNode
(node001) only.  It must be done as the user "hdfs":

\begin{verbatim}
sudo -u hdfs hdfs namenode -format
\end{verbatim}

Manually start hdfs first by starting the NameNode service on node001:

\begin{verbatim}
sudo service hadoop-hdfs-namenode start
\end{verbatim}

Then start the datanode service on node002 - node005 by typing this on
each of them one at a time:

\begin{verbatim}
sudo service hadoop-hdfs-datanode start
\end{verbatim}

Once the NameNode service is running on node001 and the DataNode service
is running on the other nodes, it is time to create new folders in HDFS
for Hadoop to use and set permissions correctly.  Note these are HDFS
paths, NOT normal Linux filesystem paths:

\begin{verbatim}
sudo -u hdfs hadoop fs -mkdir -p /tmp/hadoop-yarn/staging
sudo -u hdfs hadoop fs -chmod -R 1777 /tmp
sudo -u hdfs hadoop fs -mkdir -p /user/history
sudo -u hdfs hadoop fs -chmod -R 1777 /user/history
sudo -u hdfs hadoop fs -chown mapred:hadoop /user/history
sudo -u hdfs hadoop fs -mkdir -p /var/log/hadoop-yarn
sudo -u hdfs hadoop fs -chown yarn:mapred /var/log/hadoop-yarn
\end{verbatim}

The administrator much make some choices about where users will be storing
their files, what permissions should exist on the /user directory, etc.  If
privacy is important, the administrator must create user directories in
HDFS individually for every user on the system and set the permissions
on those directories accordingly.  For example, to create a private
storage location for users joe and bob, the adminsitrator would type:

\begin{verbatim}
sudo -u hdfs hadoop fs -mkdir /user/joe /user/bob
sudo -u hdfs hadoop fs -chown joe /user/joe
sudo -u hdfs hadoop fs -chown bob /user/bob
sudo -u hdfs hadoop fs -chmod  700 /user/joe /user/bob
\end{verbatim}

This would permit joe and only joe to read, write, and create files
in \verb|/user/joe|.  Only bob could read, write, and create files in
\verb|/user/bob|.

In a research group where all users are permitted to see files created by
all other users, the easiest approach is just to set permissions such that
anyone can create new directories inside the /user directory themselves.
To do this, the administrator would set the permissions like this:

\begin{verbatim}
sudo -u hdfs hadoop fs -chmod 1777 /user
\end{verbatim}

Then an individual user, like bob, could create his own directory where he
can store data, rather than having the adminsitrator create directories
for everyone individually.  Bob could just log into the frontend server
and type this himself:

\begin{verbatim}
hadoop fs -mkdir /user/bob
\end{verbatim}


\subsection{Starting the Hadoop Cluster}
Configuration settings and environment varaibles have been changed
such that rebooting all servers should cause all Hadoop services to
automatically start and the environment to be correctly set for all new
logins.  Do that now by rebooting node001 through node005, and the frontend
server, by typing this on each node:

\begin{verbatim}
sudo reboot
\end{verbatim}

\subsection{Checking the status of the Hadoop Cluster}
You can see the status of the ResourceManager, the jobs
that are running, have been completed, etc. by browsing to
\url{http://node001.example.com:8088}.  This may require firewall settings
to be adjusted before this is visible to your browser.  Clicking on the
Nodes link in the left sidebar should show that node002 through node005
are running and reporting in.

To see the status of HDFS, which nodes are up, how much space is used
and available, etc., browse to \url{http://node001.example.com:50070}.
This may also require firewall settings to be adjusted before this is
visible to your browser.  Clicking the Datanodes link should show that
node002 through node005 are running and reoprting in.
